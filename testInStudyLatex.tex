\documentclass[11pt,a4paper]{report}

\usepackage{ctex}
\usepackage{graphics}
\usepackage{array}
\usepackage{tabularx}
\usepackage{booktabs}

\title{ \textbf{华能山东半岛北 BW 场址海上风电项目} \\	可行性研究报告}
\author{ 中国电建集团中南勘测设计研究院}
\date{\today}


\begin{document}
	
\maketitle

\begin{abstract}
	随着高层建筑高度的增加,风荷载带来的影响越发不容忽视。本文研究了超高层建筑涡激振动的相关问题。
\end{abstract}

\section{第一章}
“你好,世界!”来自 \LaTeX{} 的问候。

This   is   a   test.

This\hspace{3em}is\hspace{3em}a\hspace{3em}test.	

在\LaTeX{}中排版中文。
汉字和English单词混排,通常不需要在中英文之间添加额外的空格。
当然,为了代码的可读性,加上汉字和 English 之间的空格也无妨。 
汉字换行时不会引入多余的空格。这还差不多,可以可以,比我想象的要好了

Several spaces equal one.
Front spaces are ignored.

An empty line starts a new
paragraph.\par
A \verb|\par| command does
the same.

This is an % short comment
% ---
% Long and organized
% comments
% ---
example: Comments do not bre%
ak a word

\# \$ \% \& \{ \} \_
\^{} \~{} \textbackslash

It's difficult to find \ldots

It's dif{}f{}icult to f{}ind \ldots

``Please press the `x' key.'' 

中文引号测试:‘你好’ “你好”

\newpage

另外需要注意的是,使用 \verb|\\|
断行命令 \\ 不会令内容另起一段,
而是在段落中直接开始新的一行。

使用 \verb|\newline| 断行的效果
\newline
与使用 \verb|\linebreak| 断行的效果
\linebreak
进行对比。

I think this is: su\-per\-cal\-%
i\-frag\-i\-lis\-tic\-ex\-pi\-%
al\-i\-do\-cious.

A reference to this subsection
\label{sec:this} looks like:
``see section~\ref{sec:this} on
page~\pageref{sec:this}.''

\footnote{需加载相关宏包,如 caption}

“天地玄黄,宇宙洪荒。日月盈昃,辰宿列张。” \footnote{出自《千字文》。 }
\newline

\begin{tabular}{l}
	\hline
	“天地玄黄,宇宙洪荒。日月盈昃,辰宿列张。” \footnotemark \\
	\hline
	\newline
\end{tabular}
\footnotetext{表格里的名句出自《千字文》。 }

\marginpar{\footnotesize 边注较窄,不要写过多文字,最好设置较小的字号。 }
	
\begin{enumerate}
	\item An item.
	\begin{enumerate}
		\item A nested item.\label{itref}
		\item[*] A starred item.
	\end{enumerate}
	\item Reference(\ref{itref}).
	\newline		
\end{enumerate}

\begin{itemize}
	\item An item.
	\begin{itemize}
		\item A nested item.
		\item[+] A `plus' item.
		\item Another item.
	\end{itemize}
	\item Go back to upper level.
	\newline
\end{itemize}

\begin{description}
	\item[Enumerate] Numbered list.
	\item[Itemize] Non-numbered list.
	\newline
\end{description}

\renewcommand{\labelitemi}{\ddag}
\renewcommand{\labelitemii}{\dag}
\begin{itemize}
	\item First item
	\begin{itemize}
		\item Subitem
		\item Subitem
	\end{itemize}
	\item Second item
\end{itemize}

\begin{center}
	Centered text using a
	\verb|center| environment.
\end{center}
\begin{flushleft}
	Left-aligned text using a
	\verb|flushleft| environment.
\end{flushleft}
\begin{flushright}
	Right-aligned text using a
	\verb|flushright| environment.
	\newline
\end{flushright}


《木兰诗》:
\begin{quotation}
	万里赴戎机,关山度若飞。
	朔气传金柝,寒光照铁衣。
	将军百战死,壮士十年归。
	归来见天子,天子坐明堂。
	策勋十二转,赏赐百千强。……
\end{quotation}

《木兰诗》:
\begin{quote}
	万里赴戎机,关山度若飞。
	朔气传金柝,寒光照铁衣。
	将军百战死,壮士十年归。
	归来见天子,天子坐明堂。
	策勋十二转,赏赐百千强。……
\end{quote}

Rabindranath Tagore's short poem:
\begin{verse}
	Beauty is truth's smile
	when she beholds her own face in
	a perfect mirror. 
	Beauty is truth's smile
	when she beholds her own face in
	a perfect mirror. 
	Beauty is truth's smile
	when she beholds her own face in
	a perfect mirror.
	Beauty is truth's smile
	when she beholds her own face in
	a perfect mirror.
\end{verse}

\begin{verbatim}
	#include <iostream>
	int main()
	{
		std::cout << "Hello, world!"
		<< std::endl;
		return 0;
	}
\end{verbatim}

\begin{verbatim*}
	for (int i=0; i<4; ++i)
	printf("Number %d\n",i);
\end{verbatim*}

\verb|\LaTeX| \\
\verb+(a || b)+ \verb*+(a || b)+

\begin{tabular}{|c|}
	center-\\ aligned \\
\end{tabular},
\begin{tabular}[t]{|c|}
	top-\\ aligned \\
\end{tabular},
\begin{tabular}[b]{|c|}
	bottom-\\ aligned\\
\end{tabular} tabulars.

\begin{tabular}{lcr|p{6em}}
	\hline
	left & center & right & par box with fixed width\\
	L & C & R & P \\
	\hline
\end{tabular}

\begin{tabular}{@{} r@{:}lr @{}}
	\hline
	1 & 1 & one \\
	11 & 3 & eleven \\
	\hline
\end{tabular}

\begin{tabular}{ r@{:}lr }
	\hline
	1 & 1 & one \\
	11 & 3 & eleven \\
	\hline
\end{tabular}

\begin{tabular}{>{\itshape}r<{*}l}
	\hline
	italic & normal \\
	column & column \\
	\hline
\end{tabular}

\begin{tabular}{>{aaa}l<{back word}c<{\hspace{1cm}}r}
	\hline
	left aligned & centered & right aligned \\
	\hline
	1 & 2 & a3 \\
	1 & 2 & a3 \\
	1 & 2 & a3 \\
	1 & 2 & a3 \\
	\hline
\end{tabular}

\begin{tabular}%
	{>{\centering\arraybackslash}p{9em}}
	\hline
	Some center-aligned long text. \\
	\hline
\end{tabular}

\newcommand\txt{a b c d e f g h i}
\begin{tabular}{cp{2em}m{2em}b{2em}}
	\hline
	pos & \txt & \txt & \txt \\
	\hline
\end{tabular}

\begin{tabular*}{14em}%
	{@{\extracolsep{\fill}}|c|c|c|c|}
	\hline
	A & B & C & D \\ \hline
	a & b & c & d \\ \hline
\end{tabular*}

\begin{tabularx}{14em}%
	{|*{4}{>{\centering\arraybackslash}X|}}
	\hline
	A & B & C & D \\ \hline
	a & b & c & d \\ \hline
\end{tabularx}

\begin{tabular}{|c|c|c|}
	\hline
	4 & 9 & 2 \\ \cline{2-3}
	3 & 5 & 7 \\ \cline{1-1}
	8 & 1 & 6 \\ \hline
\end{tabular}

\begin{tabular}{cccc}
	\toprule
	& \multicolumn{3}{c}{Numbers} \\ \cmidrule{2-4}
	& 1 & 2 & 3 \\ 	\midrule
	Alphabet & A & B & C \\
	Roman & I & II & III \\
	\bottomrule
\end{tabular}

\end{document}
















