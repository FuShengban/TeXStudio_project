\documentclass[12pt, letterpaper]{article}

\usepackage{graphicx} % Required for inserting images
\usepackage{ctex}
\usepackage{fontspec}



\title{ \textbf{华能山东半岛北 BW 场址海上风电项目} \\
	可行性研究报告}

\begin{document}
	
	\maketitle
	
	\tableofcontents
	
	\section{综合说明}
	\subsection{概述}
	\subsubsection{项目背景}
	山东省是海洋强省,海洋资源得天独厚,海上风电作为海洋经济
	的重要组成部分,在提供清洁能源,促进能源结构调整,改善国民膳
	食结构和推动供给侧结构性改革和新旧动能转换等方面具有重要意
	义。习近平总书记指出: “提高海洋资源开发能力,着力推动海洋经
	济向质量效益型转变;要保护海洋生态环境,着力推动海洋开发方式
	向循环利用型转变;坚持集约节约用海,提高海域资源使用效率。ab
	
	
	《山东省十四五能源发展规划》明确指出,到2022年,全省开工建设海上风电装机规模达到300万千瓦左右。统筹海洋能源开发利用,科学布局海上风电、滩涂光伏发电、潮流能、波浪能等海洋能发电利用项目。大力推动海洋新能源示范应用,加强配套装备研发。构建大型可燃冰开采技术仿真模拟系统,建设综合性可燃冰技术研发基地。做好全省海上风电发展规划修编工作,加快开展黄海和渤海不同类型海域离岸海上风电与海洋牧场融合发展试验,健全海上风电产业技术标准体系和用海标准,在水深超过10米、离岸10公里以外的海域科学有序开发海上风电。加强6兆瓦、8兆瓦、10兆瓦及以上大功率海上风电设备研制及使用,突破离岸独立发电装置、漂浮式载体、海底电缆、发电装置防腐蚀等关键技术。
	
	\subsubsection{项目建设意义}
	在山东省开展海上风能资源的开发和利用,是推动山东省“新旧动能转换”和“建设山东海洋强省”战略的具体行动,不但可以提供清洁能源,改善山东省能源结构,还可以推动技术创新,改善配套产业布局,促进地方经济的协调发展,同时使山东省走在践行“海洋强国战略思想”的前列,为推动海洋能源的综合开发夯实基础。
	
	\subsubsection{项目基本情况}
	华能山东半岛北BW场址位于山东省龙口市桑岛西北部附近海域,面积约为81.5km²,规划装机容量为510MW。风电场距离场址中心的离岸距离约为18km,该场区水深在14.8-16.8m之间,拟布置60台WTG230-8500型风力发电机组。为配套工程,将建设1座220kV海上升压站和1座陆上集控中心。风电机组发出的电能,通过35kV海底电缆传输至配建的220kV海上升压站,升压至220kV电压等级后通过2条220kV海底电缆接入220kV陆上集控中心。
	
	\subsection{风能资源}
	本项目位于山东烟台北部海域,本风电场安装有1座漂浮式激光雷达测风设备BW1\#。本阶段选择该测风塔对风电场进行风能资源分析。
	
	风电场风电机组轮毂高度处测风年年平均风速为7.12m/s,年平均风功率密度为370.5W/m2。根据《风电场风能资源评估方法》(GB/T 18710-2002)风功率密度等级评判标准,本风电场风功率等级为2级,风能资源丰富,具备一定的开发价值。
	
	风电场主风向和风能均主要集中在NE\~ENE及S\~SSW,其中以SSW向风向频率最大,占比10.16\%;SSW向风能频率最大,占比15.76\%。风电场风能风向分布较为分散。风速年内变化以春冬季风速相对较大,夏秋季相对偏小,风速变化幅度较大,日内变化幅度较小。
	
	风电机组轮毂高度50年一遇最大风速为36.9m/s;根据测风塔实测资料推算得V=15m/s时平均湍流强度为0.046,根据国际电工协会IEC61400-1(2019)标准,风电场机型需选择适合IEC IIIc类及以上等级的风电机组。
	
	我还应该怎么改进呢
	
\end{document}


